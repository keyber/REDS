\documentclass[12pt]{article}

\usepackage[top=3cm,bottom=3cm]{geometry}
\geometry{a4paper}

\usepackage{polyglossia}
\setdefaultlanguage{french}
\usepackage{csquotes}
\usepackage{unicode-math}
%\usepackage{fontspec}
\usepackage{xltxtra}
\usepackage{amsmath}
%\setmathfont{texgyretermes-math.otf}
\setmainfont[Ligatures=Rare]{Linux Libertine O}%

\renewcommand{\textsc}{}

\usepackage[htt]{hyphenat}
\usepackage{subcaption}

\usepackage{amsopn}
\usepackage{stmaryrd} %% llbracket
\usepackage{algorithm2e}

\usepackage{float}

\newenvironment{algorithme}[1][t]
  {\renewcommand{\algorithmcfname}{Algorithme}% Update algorithm name
   \begin{algorithm}[#1]%
  }{\end{algorithm}}


% Macros
\usepackage{xparse}
\usepackage{xifthen} % \ifthenelse

\newtheorem{theorem}{Théorème}[section]
\newtheorem{definit}{Définition}[section]
\newtheorem{corollary}{Corollaire}[theorem]
\newtheorem{lemma}[theorem]{Lemme}

\usepackage{tabularx}
\newcolumntype{Y}{>{\centering\arraybackslash}X}
\usepackage{multirow}


%%%% Imiter Mlapafr2 avec Bibtex %%%%
%% TODO: éliminer les attributs superflus pour la bibliographie
\usepackage[backend=biber,style=authoryear-comp,uniquename=init,firstinits=true,
            %% "et al" pour > deux auteurs, & pour exactement 2
            uniquelist=false,maxcitenames=2,mincitenames=1,maxbibnames=99,
            isbn=false,url=false,doi=false
]{biblatex}

\DeclareSourcemap{
  \maps[datatype=bibtex]{
    \map{
      \step[fieldsource=series,
        match={\regexp{\s}}, replace={\regexp{\\nobreakspace\x20}}]
      }
  }
}

\renewcommand{\cite}{\parencite}
\renewcommand*{\nameyeardelim}{\addcomma \addnbspace}
% \renewcommand*{\multinamedelim}{\space} % fait le contraire de ce qu'on veut
\renewcommand*{\revsdnamedelim}{}
\renewcommand*\finalnamedelim{ \& }

\DefineBibliographyExtras{french}{\restorecommand\mkbibnamelast}

\DeclareNameAlias{default}{last-first}
\DeclareNameAlias{sortname}{last-first}

% Supprimer les guillemets dans les titres
\DeclareFieldFormat[article,incollection,unpublished,inproceedings]{title}{#1}

% Je n'aime pas, mais j'ai l'impression que mlapafr utilise ça
\renewbibmacro{in:}{\printtext{In} \addspace}


% Espaces insécables dans les citations et la bibliographie (noms de
% conférences) ?

\usepackage{xpatch}
\usepackage{xstring}
%\xpatchbibmacro{series+number}{\addspace}{\addnbspace}{}{}

\renewbibmacro*{series+number}{%
  \setunit*{\addnbspace}%
  \printfield{series}%
  \printfield{number}%
  \newunit}


\xpatchbibmacro{textcite}{\addcoma}{}{}{}

\addbibresource{references.bib}

\usepackage{hyperref}
\usepackage{xspace} % Espaces après macros


\usepackage{todonotes}
\newcommand{\question}[1]{\todo[color=green!40]{#1}}
\newcommand{\questioni}[1]{\todo[color=green!40,inline]{#1}}


%%%% Macros Arthur %%%%
\newcommand{\diff}[2]{\frac{\partial{}{#1}}{\partial{}{#2}}}

\usepackage{mathtools}
\DeclarePairedDelimiterX\setc[2]{\{}{\}}{#1 \;\delimsize\vert\; #2}

\newcommand\mimic{\texttt{MIMIC-III} }
\newcommand\wordtovec{\texttt{Word2Vec} }
\newcommand\doctovec{\texttt{Doc2Vec} }
\newcommand\wordpieces{\emph{word pieces} }
\newcommand\bow{\emph{bag of words} }
\newcommand\skipgram{\emph{skip-gram} }
\newcommand\cbow{\emph{CBOW} }
\newcommand\batchnorm{\emph{batch normalization} }
\newcommand\bert{\emph{BERT} }

\newcommand{\contrainte}[1]{($\text{C}_{#1}$)}


% Notation suites mathématiques
%% 3 options - nécessite xparse !
%% \newcommand{\nsuite}[3][i][N]{({#3}_{#1})_{#1=1}^{#2}}
\DeclareDocumentCommand{\nsuite}{ O{N} O{i} m }
  {({#3}_{#2})_{#2=1}^{#1}}

% Notation pour 'la matrice X sans la colonne j'
\newcommand{\matremov}[2]{%
  %%Substitution i et j !
  \StrSubstitute{#2}{j}{\jmath}[\temp]%
  \StrSubstitute{\temp}{i}{\imath}[\tempp]%
  #1_{\hat{\tempp}
  }
}

% Cardinal
\newcommand{\card}[1]{\left\vert{#1}\right\vert}

% Intervalles
\NewDocumentCommand{\INTERVALINNARDS}{ m m }{
    #1 {,} #2
}
\NewDocumentCommand{\inter}{ s m >{\SplitArgument{1}{,}}m m o }{
    \IfBooleanTF{#1}{
        \left#2 \INTERVALINNARDS #3 \right#4
    }{
        \IfValueTF{#5}{
            #5{#2} \INTERVALINNARDS #3 #5{#4}
        }{
            #2 \INTERVALINNARDS #3 #4
        }
    }
}

%% intervalle entier
\newcommand{\ninter}[2]{\llbracket #1...#2 \rrbracket}

% Transposée

\makeatletter
\newcommand*{\transpose}{%
  {\mathpalette\@transpose{}}%
}
\newcommand*{\@transpose}[2]{%
  % #1: math style
  % #2: unused
  \raisebox{0.2em}{$\m@th#1\intercal$}%
}
\makeatother

%% FIXME : pas tout à fait la bonne notation
%% Voir ISO 80000-2 §2-15.7
%% \newcommand*{\tran}{^{\mkern-1.5mu\mathsf{T}}} ?
\newcommand{\tr}[1]{ {#1}^{\! \transpose}}

% Gras
% Précaution pour les lettres avec exposant : voir mathspec !
% FIXME : le " fait buguer \emph ???
\newcommand{\mb}[1]{{\boldsymbol{\mathbf{#1}}}}

% Distance à base de norme

\newcommand*{\newvarcmd}[2]{%
  \newcommand*{#1}[2][]{%
    \begingroup % \sizel and \sizer are local:
      \let\varl\left
      \let\varr\right
      \ifthenelse{\isempty{##1}}{%
        \let\sizel\relax
        \let\sizer\relax
      }{%
        \expandafter\let\expandafter\sizel\csname ##1l\endcsname
        \expandafter\let\expandafter\sizer\csname ##1r\endcsname
      }%
      #2%
    \endgroup
  }
}

\newvarcmd{\abs}{\sizel\lvert #2\sizer\rvert}
\newvarcmd{\norm}{\sizel\lVert #2\sizer\rVert}
\newcommand{\dist}[2]{\norm{#1 - #2}}

%% Macro pour les vecteurs directeurs
\NewDocumentCommand{\vdir}{ m O{j} O{r} }{\mb{#1}_{#3#2}}
\newvarcmd{\primabs}{\sizel\lvert #2\sizer\rvert}

\renewcommand{\eqref}[1]{équation~\ref{#1}}
\newcommand{\algref}[1]{algorithme~\ref{#1}}
\newcommand{\figref}[1]{figure~\ref{#1}}
\newcommand{\tabref}[1]{tableau~\ref{#1}}
\newcommand{\secref}[1]{section~\ref{#1}}

\AtBeginDocument{ %Bizarrerie unicode-math
  \DeclareMathOperator{\mmin}{\mathrm{min}}
  \DeclareMathOperator{\mmax}{\mathrm{max}}
  \DeclareMathOperator{\eexp}{\mathrm{exp}}
  \DeclareMathOperator{\argmin}{\text{argmin}}
  \newcommand{\gargmin}[2]{\argmin_{#1}\left\{{#2}\right\} }
  \DeclareMathOperator{\prox}{\mathrm{prox}}
  \newcommand{\proxg}[3]{\prox_{\frac{#1}{#2}{#3}}}
}

\begin{document}
\title{REDS - TP1 : Analyse préliminaire des données}
\author{Laura Nguyen et Keyvan Beroukhim}
\maketitle

La base de données issue du \emph{ATLAS Higgs Boson Machine Learning Challenge
2014} est constituée de 818238 évènements simulés pouvant être soit des
collisions \emph{"Higgs to tautau"}, soit du \emph{background}. Étant données
ces deux classes, \emph{"s"} pour \emph{signal} et \emph{"b"} pour
\emph{background}, l'objectif est de classifier au mieux les évènements.

\begin{figure}[H]
    \center
    \includegraphics[width=\textwidth]{images/dataset_sample.png}
    \caption{Échantillon de 5 évènements du dataset avec uniquement les 8
    premiers attributs affichés}
\end{figure}

Chaque évènement est défini par 35 attributs, dont son label. Nous retirons du
dataset les features servant uniquement pour le challenge Kaggle :
\texttt{Weight}, \texttt{KaggleSet} et \texttt{KaggleWeight}.

Les données sont composées à 34\% de signaux et à 66\% de background. Nous
remarquons cependant que le dataset contient, à vue d'œil, un nombre important
d'exemples avec des valeurs manquantes (indiquées par -999). Or, travailler avec
des datasets contenant des données manquantes est compliqué. Nous décidons donc
de supprimer les évènements dont au moins un attribut n'est pas valide et nous
nous retrouvons avec seulement 223574 exemples : restreindre le dataset aux
données complètes fait perdre 75\% des données. De plus, la base restreinte
contient 47\% de signaux, soit 13\% de plus que dans le dataset original. Si ce
dernier est représentatif de la réalité alors celui restreint ne l'est pas.


\begin{figure}[H]
    \center
    \includegraphics[width=\textwidth]{images/heatmap.png}
    \caption{Matrice de corrélation des attributs en ignorant les valeurs invalides}
    \label{img:heatmap}
\end{figure}

La variable \texttt{EventId} ne fournit pas d'information : cela se remarque,
par exemple, par une corrélation nulle avec toutes les autres variables. Par la
suite, nous ne considérerons donc pas cette variable.


Certaines variables sont fortement corrélées, d'autres anti-corrélées, e.g.
\texttt{DER\_mass\_MMC} et \texttt{DER\_mass\_transverse\_met\_lep}. Ces
informations pourront se révéler utiles si nous souhaitons traiter les évènements à
données manquantes.

La matrice de corrélation obtenue en ignorant les valeurs invalides
et celle obtenue en ignorant les évènements ayant au moins un attribut à valeur
invalide sont très similaires.

\begin{figure}[H]
    \center
    \includegraphics[width=\textwidth]{images/table_corr.png}
    \caption{Coefficients de corrélation entre chaque attribut et la classe
    \texttt{Label}}
    \label{img:table-corr}
\end{figure}

\begin{figure}[H]
    \center
    \includegraphics[width=\textwidth]{images/histogrammes.png}
    \caption{Histogramme des valeurs de chaque attribut du dataset où les
    valeurs manquantes sont ignorées}
    \label{img:hist}
\end{figure}

Plusieurs histogrammes sont très déséquilibrés, cela est peut-être dû à une
présence d'erreur dans les données. Nous pourrions essayer de traiter ces cas-là
en les supprimant par exemple.

De nombreux histogrammes présentent une répartition exponentielle décroissante
des données.

Preprocessing : transformation (non linéaire) de certaines variables en les
remplaçant par leur logpq

\end{document}
